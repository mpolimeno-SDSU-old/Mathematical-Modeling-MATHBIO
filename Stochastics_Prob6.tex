From Problem 5b, we have
$$
E(t) = \sum_{j=0}^{\infty} (N_{0}+j)P_{N_{0}+j}(t) = N_{0}e^{\lambda t}
$$
We have $N_{0}=10,000$ and that the total number of births occurred in 20 days is 4,500.\\
So, assuming that no deaths will occur in this period since this is a birth-only stochastics process, we have that our expected population at $t=20$ is
$$
E(20) = 4,500 + 10,000 = 14,500.
$$
Therefore
$$
14,500 = 10,000e^{\lambda 20},
$$
from which we can easily solve for $\lambda$
$$
\lambda = \frac{\ln \left(\frac{14,500}{10,000}\right)}{20}
$$
from which we obtain
$$
\lambda \approx 0.0186
$$
Therefore the population has a growth rate of 1.86\%.