\section{Part a}
$P_{n}$ is a vector that varies discretely with time. Let $L$ be our Leslie Matrix with $P_{n} = [P_{1}, P_{2}, P_{3}, P_{4}]_{n}^{T}$ the population vector at time $n$ divided in 4 age classes (0-1,1-2,2-3,3-4). Then the Leslie Model is
$$
P_{n+1}=LP_{n},
$$
where $L$ is our Leslie matrix defined as
\begin{align}
L  &= \begin{bmatrix}
0 & 1.5 & 2.2 & 3.4 \\
0.4 & 0 & 0 & 0 \\
0 & 0.7 & 0 & 0\\
0 & 0 & 0.75 & 0
\end{bmatrix}.
\end{align}
To find the steady-state percentage for each age group with build a model and run it. The dominant eigenvalue is found to be $\lambda$=1.2465, which implies that the population growth is about 25\% over a 1-year period. The associated eigenvector is found to be\\
\begin{align}
\textbf{$P_{e}$} &= \begin{bmatrix}
0.6213\\
0.1994\\
0.1120\\
0.0674
\end{bmatrix},
\end{align}
this shows that the steady-state distribution should be approximately 62\% in the 0-1 age group, about 20\% in the 1-2 age group, about 11\% in the 2-3 age group and finally a little less than 7\% in the 3-4 age group.\\
The time that it takes for the population to double after reaching its steady-state distribution is found to be
$$
t=3.1460,
$$
thus this population, according to this model, will double in size after roughly 3 years and two months.

\section{Part b}
The survival rates $s_{2}$ and $s_{4}$ is reduced by a fraction $\alpha$, so that the survival rate of 1-2 years old is 0.7$\alpha$ and the survival rate of 2-3 years old is 0.75$\alpha$. We compute the value of $\alpha$ in MatLab and we find it to be
$$
\alpha = 0.4324.
$$
Then, for this value of $\alpha$ and 550 mature (3-4 years old) animals, we determine the total population to be
$$
P_{tot} = 20066,
$$
whereas the population for each age group is found to be
\begin{align}
\textbf{$P_{e}$} &= \begin{bmatrix}
12860\\
5144\\
1557\\
505
\end{bmatrix}.
\end{align}
As for the total number of animals harvested each year under this conditions, this is found to be roughly equal to 834.