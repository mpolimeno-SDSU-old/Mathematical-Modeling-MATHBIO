The transition matrix for red squirrels, gray squirrels, both or neither in that order is given by
\begin{align*}
T  &= \begin{bmatrix}
    0.8797 & 0.0382 & 0.0527 & 0.0008 \\
    0.0212 & 0.8002 & 0.0041& 0.0143 \\
    0.0981 & 0.0273 & 0.8802 & 0.0527\\
    0.0010 & 0.1343 & 0.0630 & 0.9322
 	  \end{bmatrix}
\end{align*}

The equilibrium distribution of squirrels based on this transition matrix is given by the normalized eigenvector associated with the eigenvalue $\lambda=1$, which is found to be
\begin{align*}
    \textbf{$x_{e}$} &= \begin{bmatrix}
           0.1705 \\
           0.0560 \\
           0.3421 \\
           0.4313
         \end{bmatrix}
\end{align*}

This eigenvector shows that the predicted squirrel community for the given region of the Great Britain should be approximately 17\% red squirrels, 6\% grey squirrels, 34\% both and 43\% neither. Therefore the native red squirrel, which starts off as being 87\% of the total squirrel population, will end up being 17\%, while the grey squirrel goes from 2\% to 6\%. Interestingly, the total population at the equilibrium will be made of 43\% different species of squirrels, whereas at $t=0$ those were only 0.1\% of the total.\\
Therefore, this model seems to suggest that the invasive grey species will displace the native red squirrel, however the latter will not go extinct nor will be outnumbered by the invasive grey squirrel, as we can see that, there is going to be a higher overall percentage of red squirrels anyway at the equilibrium distribution.